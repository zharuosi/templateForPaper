\documentclass[onecolumn,11pt]{report}
\usepackage{longtable,graphicx}
\usepackage{nomencl}
\usepackage{amsbsy}
\usepackage[none]{hyphenat}
\usepackage[firstpage]{draftwatermark}
\SetWatermarkScale{1.0}
\SetWatermarkLightness{0.85}
\usepackage{appendix}

\usepackage{graphicx}

\usepackage{epsfig} %% for loading postscript figures
%\usepackage[outdir=./]{epstopdf}

\usepackage{graphicx}
\usepackage{float}
\usepackage{lipsum} % for dummy text

%\usepackage{titlesec}
%\usepackage{cite}
\usepackage{float}
\usepackage{color}
\usepackage[utf8]{inputenc}
%\usepackage[portuguese]{babel}
\usepackage{makeidx}
\usepackage{enumerate}
\usepackage[official]{eurosym}
\usepackage{indentfirst}

%\usepackage{hyperref}
%make page number, not text, be link on toc,lof,lot
\usepackage[linktocpage]{hyperref}
%Break the long titles, but lose the hyerlink
%\usepackage[breaklinks=true]{hyperref}
%Control the space between number and title
%\usepackage{tocbasic}
%\DeclareTOCStyleEntry[dynnumwidth]{tocline}{figure}
%\DeclareTOCStyleEntry[dynnumwidth]{tocline}{table}

\usepackage{listings}

\usepackage{array}
\usepackage{makecell}
\usepackage{multicol}
\usepackage{multirow}
\usepackage{booktabs}

\usepackage{amsmath}
\usepackage{tabularx}
\usepackage{flexisym} 
\usepackage{placeins} 

\usepackage{textcomp}

%%\usepackage{nopageno}

\usepackage{caption}
%\usepackage{subcaption}

\usepackage{siunitx}

\usepackage{appendix}

\usepackage{subfig}
\usepackage{adjustbox}


\usepackage{tabu}
\usepackage{xcolor}
\usepackage{color, colortbl}

\usepackage{dirtree}

%\usepackage{rotating}

\definecolor{dkgreen}{rgb}{0,0.6,0}
\definecolor{gray}{rgb}{0.5,0.5,0.5}
\definecolor{mauve}{rgb}{0.58,0,0.82}

\lstset{frame=tb,
  language=C,
  aboveskip=3mm,
  belowskip=3mm,
  showstringspaces=false,
  columns=flexible,
  basicstyle={\small\ttfamily},
  numbers=none,
  numberstyle=\tiny\color{gray},
  keywordstyle=\color{blue},
  commentstyle=\color{dkgreen},
  stringstyle=\color{mauve},
  breaklines=true,
  breakatwhitespace=true,
  tabsize=3,
  moredelim=**[is][\color{red}]{@}{@},
  moredelim=**[is][\color{blue}]{^}{^}
}


%\usepackage{tocloft}
%\setlength{\cftfignumwidth}{3.55em}

%\usepackage{tocloft}
%\setlength{\cftfignumwidth}{5em}

%\usepackage{caption} 
%\captionsetup[table]{skip=10pt}
%
%\usepackage{titling}
%\setlength{\columnsep}{10mm}
\makeglossary
% Define some macros:

%--------------------------------------------------------------
%\setlength{\parskip}{2ex plus0.5ex minus0.5ex}
%\setlength{\parindent}{0.0mm}
%\setlength{\textheight}{8.5in}
%\setlength{\textwidth}{6.0in}
%\setlength{\oddsidemargin}{0.5in}
%\setlength{\evensidemargin}{0.0in}
%\setlength{\topmargin}{0.0in}
%\setlength{\arrayrulewidth}{.3mm}
%--------------------------------------------------------------
\parskip .3cm
%\baselineskip 14pt
\parindent 0. in
\textheight=8.5 in
\textwidth=6.5 in
%\hoffset -0.5in
%\voffset -0.90in
\voffset .0 in
\oddsidemargin 0. in \evensidemargin 0 in \topmargin 0.0in
\renewcommand{\baselinestretch}{1}
%-------------------------------------------------------


\newcommand{\mychapter}[2]{
    \setcounter{chapter}{#1}
    \setcounter{section}{0}
    \chapter*{#2}
    \addcontentsline{toc}{chapter}{#2}
}

%----------------------------------------------------------------------------------------

\begin{document}

%%\maketitle % Print the title


\begin{titlepage}

~~
\vspace{1cm}
\centering
{\Large\bfseries {Determination of Maneuvering Coefficients for a Destroyer Model with OpenFOAM}}

\vspace{2cm}

{\large {\textit {\bf Progress Report - 1}}}

\vspace{2cm}

{\large  Shanqin Jin, Ruosi Zha and Wei Qiu}

\vspace{0.5cm}

{\large Department of Ocean and Naval Architectural Engineering \\ Memorial University}


\vspace{2cm}


{\large \underline{Submitted to}}

\vspace{1cm}

{\large Dr. Kevin McTaggart}

\vspace{0.5cm}

{\large Defence Research and Development Canada, Atlantic Research Centre}

\vspace{3cm}


{\large September 23, 2019}

\clearpage

\tableofcontents

\listoffigures
 
\listoftables




\makenomenclature
\mbox{}
\nomenclature{LE}{Leading edge}
\nomenclature{$c$}{Chord length}
\nomenclature{$Re$}{Reynolds number}
\nomenclature{$\sigma$}{Cavitation number}
\nomenclature{$\alpha$}{Angle of attack}
\nomenclature{$p_a$}{Air pressure}
\nomenclature{$p_v$}{Vapor pressure of water}
\nomenclature{$\mu$}{dynamic viscosity of water}
\nomenclature{$\nu$}{kinematic viscosity of water}
\nomenclature{$k$}{turbulence kinetic energy}
\nomenclature{$\epsilon$}{turbulence dissipation rate}
\nomenclature{$\omega$}{specific turbulence dissipation rate}
\nomenclature{$R$}{Radius of circular domain}
\nomenclature{$t$}{Maximum thickness of a foil section}
\nomenclature{$f$}{Maximum camber of a foil section}
\nomenclature{$y^+$}{Non-dimensional first-grid spacing}
\nomenclature{AR}{Aspect ratio of a structured grid}
\nomenclature{SR}{Stretching ratio}
\nomenclature{RANS}{Reynolds averaged Navier-Stokes equations}
\nomenclature{$C_{p_{\rm min}}$}{Minimum pressure coefficient}
\nomenclature{$F_d$}{Drag force}
\nomenclature{$F_l$}{Lift force}
\nomenclature{$C_d$}{Drag coefficient}
\nomenclature{$C_l$}{Lift coefficient}

\printnomenclature

\end{titlepage}



%----------------------------------------------------------------------------------------
%	INTRODUCTION
%----------------------------------------------------------------------------------------


\clearpage

\section*{\centering Executive Summary}
\addcontentsline{toc}{section}{Executive Summary}
 
Propellers are typically first rough machined on CNC milling machines. Final surfaces are finished by robotic or manual grinding. Blade edges and tips are normally made to conform to templates of their required forms using manual grinding. Manufacturing tolerances for new ship propellers are specified by, for example, ISO 484 standards, in which Class S denotes very high accuracy. Manual grinding of propeller surfaces may introduce inaccuracies and deviations from design, which are denoted as defects in the report. There is a need to investigate if these defects could lead to degradation of propeller cavitation (and hence noise) and efficiency. 

In the project led by the Canadian propeller manufacturer, Dominis Engineering, the investigation involves 2-D CFD simulations of foils without and with defects in infinite flow, 3-D CFD and experimental studies of finite-span foils and single propeller blades in various scales in a cavitation tunnel, and 3-D CFD simulations of propellers in different scales. This report summarizes 2-D CFD studies on foils without and with leading edge (LE) defects.  
 
All the LE defects examined are within ISO 484 Class S tolerances (+/-0.5mm for a 1-part template or +/-0.25mm for the 3-part template). The DTMB modified NACA66 (a=0.8 camber) sections without and with LE defects were investigated at various angles of attack using the 2-D steady RANS solver in Star-CCM+ on structured grids.

Convergence studies were first carried out to examine the effects of domain type, domain size, grid distribution, grid resolution, and turbulence model on the solution. Based on the results of convergence studies, best-practice settings were proposed for simulations of 2-D foils using Star-CCM+. With the best-practice settings, studies were carried out to verify the minimum pressure coefficient envelops of a modified NACA66  foil section ($a$=0.8, $t/c$=0.2 and $f/c$=0.02) without defect. Numerical results were generally in good agreement with the potential-flow solutions by Brockett (1966) and the RANS solutions with ANSYS CFX by Defence Research and Development Canada (DRDC) (Hally, 2018). 

CFD simulations with best-practice settings were then extended to the modified NACA66 sections ($a$=0.8, $t/c$=0.0416 and $f/c$=0.014) with three different sizes of defects near LE, representing three levels of manufacturing tolerances within Class S. The predicted minimum pressure coefficients for foils without and with LE defects were compared at various angles of attack. Comparative studies showed that the LE manufacturing defects in various sizes within the ISO Class S have significant effects on the cavitation performance of 2-D foils in terms of reduced cavitation inception speed in the typical design range of angle of attack. 

The following conclusions are made from the 2-D studies:

\begin{itemize}

\item Class S defects close to LE narrow the cavitation buckets in the $C_{p_{\rm min}}-\alpha$ space in the typical design range of angle of attack, $-1.5^\circ<\alpha<2^\circ$. As a consequence, such a defective section would experience cavitation at a lower speed than the design one. Smaller defects than Class S maximum deviation show a similar effect.

\item The detrimental cavitation effect seems to be primarily on the side of the section with defect. A defect right on the leading edge ($x$=0 and $y$=0) would affect cavitation on both sides of the section. 

\item The defects can cause pressure drops at the furthest-forward edge of a LE defect. This leads to flow separation at angles of attack at the upper end of normal range of operations. When a section experiences flow separations, the section lift decreases and the drag increases resulting in a reduction in propeller efficiency.

\end{itemize}



\newpage

\chapter{Introduction}

Underwater radiated noise (URN) from ships is being recognized as a world-wide problem since underwater noise from shipping is increasingly being considered as a significant and omnipresent pollutant with the potential to impact marine ecosystems on a global scale. Continued growth in the number of ships will significantly increase the total volume of noise generated by the global shipping fleet. Projections suggest that URN level could increase by as much as a factor of 1.9 of the current level by the year 2030 (Southall et al., 2017). The URN of a ship is caused mainly by the propeller and the main machinery. The European Union's collaborative research project AQUO (Achieve QUieter Oceans) has provided valuable insight into the relative contribution of each source of noise generated by different types of ships (AQUO, 2015). A significant conclusion of the study is that for ferries and cruise vessels at normal operating speeds, propeller cavitation is the most important source of noise. The noise levels from a ship jump substantially when propeller cavitation begins. 

Many studies have been carried out to investigate effects of design parameters on cavitation and efficiency performance of a propeller with an objective to avoid or control vortex cavitation and to improve its efficiency. However, very little effort has been made to understand the impact of propeller manufacturing tolerances on the propeller performance, and no paper in the public literature was found to address this aspect. A preliminary CFD study carried out by Hally (2018) indicates that the manufacturing defects potentially have large impact on propeller cavitation performance.

The majority of propellers manufactured today are hand- or robotic finished from castings which are rough machined using Computer Numerical Control (CNC) (Van Beek and Janssen, 2000, Janssen and Leever, 2017). Blade edges and tips, the most sensitive parts of the geometry of a propeller,  are made to conform to templates of their required form using manual grinding. Manual grinding of propeller surfaces introduces inaccuracies and deviations from design, which could lead to degradation of propeller performance in terms of efficiency, cavitation and noise. 

Manufacturing tolerances for new ship propellers are specified by organizations, such as International Standards Organization (ISO), which defines the manufacturing standards for propeller construction, and the Naval Sea Systems Command, USA (NAVSEA), which provides manufacturing standards for US Navy's ship construction (2004). The ISO 484-1 and ISO 484-2  standards for manufacturing tolerances for ship propellers (2005) were established in 1981 by adopting an ISO Recommendation of 1966. ISO 484-1 is applicable to propellers with diameters greater than 2.5m, while ISO 484-2 is applicable to propellers with diameters from 0.8m to 2.5m. There are four classes of tolerances in each standard. Each tolerance class is intended for a certain type of vessels. Among the four classes, Class S denotes the smallest tolerance and hence the highest accuracy. 

Commercial ship screw propellers are typically manufactured to meet the ISO 484 Standard (ISO, 2015). In a recent workshop, hosted by CISMaRT and Transport Canada, on ship noise mitigation technologies, it was suggested to study the effect of manufacturing defects on propeller cavitation performance with an objective to reduce URN (CISMaRT, 2018).  

Canadian propeller manufacturer Dominis Engineering has been at the forefront of propeller manufacturing since its formation in 1985. Over the years the company has developed state-of-the-art technology for manufacturing of propellers and propeller blades, (1, 2). The guiding principle of Dominis approach to propeller manufacturing is the complete elimination of both robotic and hand grinding, i.e. the propeller should be completely finished on the CNC milling machine. Implementation of this principle has a benefit of greatly improving the accuracy of propellers manufactured by Dominis The precision routinely achieved using Dominis technology is significantly better than class S ISO 484-1 (3, 4).

Propellers for commercial ships are typically manufactured to meet the ISO 484-1 standard. At a recent workshop, hosted by CIMaRT and Transport Canada, on ship noise mitigation technologies the impact of propeller manufacturing tolerances and manufacturing defects on cavitation performance was discussed. In partnership with Transport Canada, Memorial University and DRDC-Atlantic, Dominis Engineering initiated a study to determine the impact LE defects have on propeller performance. The project consists of three elements:

\begin{itemize}

\item 2-D CFD simulations of foils in infinite flow.\\
\item 3-D CFD and experimental studies of a finite-span foil and single propeller blades in various scales in a cavitation tunnel. \\
\item 3-D CFD simulations of propellers in different scales.

\end{itemize}

This report summarizes the 2-D CFD studies on foil sections without and with LE defects. 


\section{Scope of the Present Work}

The objective of this project is to study the cavitation performance of an ideal propeller blade section (as designed) and an “as-built” propeller blade section, which are within the limits of ISO 484-1 Class S tolerances (highest precision in the standard). The investigation into propeller manufacturing tolerances was carried out using CFD simulations with a steady RANS solver on structured grids. Small geometric variations relative to the dimensions of the propeller suggest that much could be learned from less computationally intensive simulations based on 2D sections before full-scale 3D propeller simulations. 

Extensive simulations in this project were focused on examining the effect of LE defects of 2-D modified NACA66 foil sections on their cavitation performance. 

Convergence studies were first carried out for foils without and with defects using rectangular and circular computational domains. Effects of simulation parameters, such as domain size, grid resolution, grid distribution, grid stretching ratio, grid aspect ratio, first-grid spacing, $y^+$, and turbulence model, on the solution were carefully examined. Turbulence models considered in this work include four eddy viscosity models: Spalart-Allmaras, $k-\epsilon$, $k-\omega$, and SST $k-\omega$ models, and two Reynolds stress models: the elliptic bending model, EB-RSM and the linear pressure-strain two-layer model, LPS-RSM. A total of 1,020 cases have been simulated. Based on the results of convergence studies, the best-practice settings for 2-D simulations with  the steady RANS solver in Star-CCM+ were proposed. 

Using the best-practice settings, verification studies were carried out for the cavitation buckets of a modified NACA66 section ($a$=0.8, $t/c$=0.2, $f/c$=0.02) without defect by comparing the RANS results with the potential-flow solutions by Brockett (1966) and the RANS solutions with ANSYS CFX (Hally, 2018). Note that $t$, $f$ and $c$ denote the maximum thickness, the maxium camber and the chord length of a foil section, respectively. Furthermore, the minimum pressure coefficients for the modified NACA66 sections ($a$=0.8, $t/c$=0.0416, $f/c$=0.014) with three different sizes of defects near LE, representing three levels of manufacturing tolerances within Class S, were compared at various angles of attack.


%----------------------------------------------------------------------------------------%

\chapter{Theoretical Background of Star-CCM+}



\section{Governing Equations}

The governing RANS equations for the incompressible viscous flow are:
\begin{equation}
\frac{\partial u_i}{\partial x_i} = 0
\label{eq1}
\end{equation}  
\begin{equation}
\rho \frac{\partial u_i}{\partial t} + \rho u_j \frac{\partial u_i}{\partial x_j} = -\frac{\partial p}{\partial x_i} + \frac{\partial}{\partial x_j} \left[ \mu \left(\frac{\partial u_i}{\partial x_j} + \frac{\partial u_j}{\partial x_i} \right) \right] + \frac{\partial}{\partial x_j} (-\rho \overline{u'_i u'_j})
\label{eq2}
\end{equation}

where $u_i$, $i=1,2$ denotes the velocity components along $x$- and $y$-axis, respectively, for a two-dimensional flow, $p$ is the pressure, $\rho$ is the density of water, $\mu$ is the dynamic viscosity of water, and $-\rho \overline{u'_i u'_j}$ are the Reynolds stresses. 

\section{Turbulence Modeling}

The Reynolds stresses can be solved based on the Boussinesq hypothesis using the eddy viscosity turbulence models, or be solved from the transport equation based on Reynolds stress models. In the present studies, one-equation and two-equation eddy viscosity models as well as Reynolds stress models were employed to solve the RANS equations.

In the eddy viscosity models, it is assumed that the Reynolds stresses are related to the mean velocity gradients, the turbulence kinetic energy and the eddy viscosity, i.e.,
\begin{equation}
-\rho \overline{u'_i u'_j} = \mu_t \left( \frac{\partial u_i}{\partial x_j} + \frac{\partial u_j}{\partial x_i} \right) - \frac{2}{3}\rho k \delta_{ij}
\label{eq3}
\end{equation}

where $\mu_t$ represents the eddy viscosity, $\delta_{ij}$ is the Kronecker delta, $k=\frac{1}{2}\overline{u'_i u'_j}$ is the turbulent kinetic energy that can be solved from the transport equations. The Reynolds stress tensor is linearly proportional to the mean strain rate.  

The one-equation Spalart-Allmaras (SA) model solves a transport equation for the modified diffusivity, $\tilde{\nu}$, to determine the turbulence eddy viscosity, $\mu_t$, i.e.,
\begin{equation}
\mu_t=\rho f_{\nu 1} \tilde{\nu}
\label{eq4}
\end{equation}

where $f_{\nu 1}$ is a damping function. The transport equation for the modified diffusivity is:
\begin{equation}
\frac{\partial}{\partial t} (\rho \tilde{\nu}) + \nabla \cdot \left (\rho \tilde{\nu} \bar{\textbf{v}} \right ) = \frac{1}{\sigma_{\tilde{\nu}}} \nabla \cdot \left [ (\mu + \rho \tilde{\nu}) \nabla \tilde{\nu} \right ] + P_{\tilde{\nu}} + S_{\tilde{\nu}}
\label{eq5}
\end{equation}

where $\bar{\textbf{v}}$ is the mean velocity, $\sigma_{\tilde{\nu}}$ is a model coefficient, $\mu$ is the dynamic viscosity, $P_{\tilde{\nu}}$ is the production term, and $S_{\tilde{\nu}}$ is the source term. The SA model has good convergence and robustness for specific flows. However, the turbulence length and time scales are not well defined as they are in other two-equation models.

Two-equation models are widely used to solve the RANS equations, in which both the velocity and length scale are solved using separate transport equations. The turbulence length scale is estimated from the kinetic energy and its dissipation rate. The standard $k-\epsilon$ model,  the standard $k-\omega$ model and the Shear Stress Transport (SST) $k-\omega$ models were investigated in the present work. 

In the $k-\epsilon$ model, the turbulent eddy viscosity is calculated as:
\begin{equation}
\mu_t=\rho C_{\mu}f_{\mu} kT
\label{eq6}
\end{equation}

where $C_{\mu}$ is a model coefficient, $f_{\mu}$ is a damping function,  and $T$ is the turbulent time scale calculated by:
\begin{equation}
T = \max(T_e,C_t\sqrt{\frac{\nu}{\epsilon}})
\label{eq7}
\end{equation}

where $T_e=\frac{k}{\epsilon}$ is the large-eddy time scale, $C_t$ is a model coefficient, $\nu$ is the kinematic viscosity. The transport equations for the turbulent kinetic energy, $k$, and the turbulence dissipation rate, $\epsilon$, are written as:
\begin{equation}
\frac{\partial}{\partial t} (\rho k) + \nabla \cdot (\rho k \bar{\textbf{v}}) = \nabla \cdot \left[ (\mu + \frac{\mu_t}{\sigma_k}) \nabla k \right] + P_k -\rho(\epsilon-\epsilon_0) + S_k
\label{eq8}
\end{equation}

\begin{equation}
\frac{\partial}{\partial t} (\rho \epsilon) + \nabla \cdot (\rho \epsilon \bar{\textbf{v}}) = \nabla \cdot \left[ (\mu + \frac{\mu_t}{\sigma_{\epsilon}}) \nabla \epsilon \right] + \frac{1}{T_e} C_{\epsilon 1} P_{\epsilon} - C_{\epsilon 2} f_2 \rho(\frac{\epsilon}{T_e}-\frac{\epsilon_0}{T_0}) + S_{\epsilon}
\label{eq9}
\end{equation}

where $\sigma_k$, $\sigma_{\epsilon}$, $C_{\epsilon 1}$ and $C_{\epsilon 2}$ are the model coefficients, $P_k$ and $P_{\epsilon}$ are the production terms, $f_2$ is a damping function, and $S_k$ and $S_{\epsilon}$ are the source terms. 

In the realizable $k-\epsilon$ model, the equation for turbulence dissipation rate is modified and the coefficient $C_{\mu}$ is expressed as a function of mean flow and turbulence properties instead of constant in the standard model. The effect of the mean flow distortion on turbulent dissipation is introduced to improve the performance for rotation and streamline curvature. It also improves the boundary layer under strong adverse pressure gradients or separation. 

The renormalization group (RNG) $k-\epsilon$ model is based on  the renormalization group analysis of the Navier-Stokes equations. Different constants are used in the transportation equations for the turbulence kinetic energy and dissipation. The RNG $k-\epsilon$ model leads to lower turbulence levels and less viscous flows. 

In the $k-\omega$ model, the turbulent eddy viscosity is related to the turbulence kinetic energy, $k$, and the specific turbulence dissipation rate, $\omega$, which is also referred to the mean frequency of the turbulence. The turbulent eddy viscosity is calculated as:
\begin{equation}
\mu_t=\rho kT
\label{eq10}
\end{equation}

where $T=\alpha^*/\omega$ is the turbulence time scale in the standard $k-\omega$ model and $\alpha^*$ is a model coefficient. The transport equations for the turbulent kinetic energy, $k$, and the specific dissipation rate, $\omega$, are written as:
\begin{equation}
\frac{\partial}{\partial t} (\rho k) + \nabla \cdot (\rho k \bar{\textbf{v}}) = \nabla \cdot \left[ (\mu + \sigma_k \mu_t) \nabla k \right] + P_k -\rho \beta^*f_{\beta^*}(\omega k - \omega_0 k_0) + S_k
\label{eq11}
\end{equation}

\begin{equation}
\frac{\partial}{\partial t} (\rho \omega) + \nabla \cdot (\rho \omega \bar{\textbf{v}}) = \nabla \cdot \left[ (\mu + \sigma_\omega \mu_t) \nabla \omega \right] + P_\omega -\rho \beta f_{\beta}(\omega^2 - \omega_0^2) + S_k
\label{eq12}
\end{equation}

where $\sigma_k$ and $\sigma_{\omega}$ are model coefficients, $P_k$ and $P_{\omega}$ are production terms, and $f_{\beta^*}$ is the free-shear modification factor or the vortex-stretching modification factor, $k_0$ and $\omega_0$ are the ambient turbulence values that counteract turbulence decay, and $S_k$ and $S_{\omega}$ are the source terms. The $k-\omega$ model predicts strong vortices and the near-wall interactions more accurately than the $k-\epsilon$ models. The limitations of the original $k-\omega$ model include the over-prediction of shear stresses of adverse pressure gradient boundary layers, and the sensitivity to initial conditions and inlet boundary conditions. 

For the SST $k-\omega$ model, the transport equations are the same as those of the standard $k-\omega$ model when setting the damped cross-diffusion derivative term as zero in the near field. In the far field, the transport equations are the same as those of the standard $k-\epsilon$ model, which can avoid the problem that the model is too sensitive to the inlet turbulence properties. Detailed formulations can be found in the work by Menter (1993). The SST $k-\omega$ model introduces the transport of the turbulence shear stress and improves the prediction of the onset and the flow separation under adverse pressure gradients.

In the Reynolds stress models (RSM), the transport equations are solved for all the components of the Reynolds stress tensor and the turbulence dissipation rate, i.e.,
\begin{equation}
\frac{\partial}{\partial t} (\rho \overline{u'_i u'_j}) + \frac{\partial}{\partial x_k} (\rho u_k \overline{u'_i u'_j} ) = P_{ij} + F_{ij} + D_{ij}^T + \phi_{ij} - \epsilon_{ij}
\label{eq13}
\end{equation}

where $P_{ij}$ is the stress production, $F_{ij}$ is the rotation production, $D_{ij}^T$ is the turbulent diffusion, $\phi_{ij}$ is the pressure strain tensor, and $\epsilon_{ij}$ is the dissipation rate tensor. The isotropic turbulent dissipation rate is solved from a transport equation analogous to the $k-\epsilon$ model with various model coefficients. 

In order to resolve the viscous sub-layer, two RSM models are implemented in Star-CCM+, including the elliptic blending model (EB-RSM) and the linear pressure-strain two-layer model (LPS-RSM). These two models were studied in the present work. The EB-RSM model applies only one scalar elliptic equation instead of the original six transport equations for all stress components, which is based on the relaxation formulations of the pressure-strain tensor using a blending function. In the LPS-RSM model, the pressure-strain term, $\phi_{ij}$,  comprises a slow term, also known as the return-to-isotropy term, a rapid term, and wall-reflection terms. In general, the Reynolds stress models can predict complex flows with swirl rotation and high strain rates more accurately than eddy viscosity models. 
%----------------------------------------------------------------------------------------%
 

\chapter{Numerical Simulations}

Numerical simulations were carried out using the steady RANS solver in Star-CCM+ on structured grids for the DTMB modified NACA66 foils ($a$=0.8, $t/c$=0.0416, $f/c$=0.014) without and with LE defects in an infinite domain.  

In addition, simulations were carried out to the DTMB modified NACA66 foils ($a$=0.8, $t/c$=0.2, $f/c$=0.02) without defect with an objective to verify the numerical results by comparing them with the potential-flow solutions and the RANS results with ANSYS CFX.  

In the present work, the chosen modified NACA66 foils have the chord length of 1,000mm. The inflow velocity is 30 m/s in full scale.

\section{LE Manufacturing Defects}

A LE defect is determined by measuring the difference between the template and the manufactured blade section. According to ISO 484 (2005), Class S tolerances are defined as +/-0.5mm for a 1-part template or +/-0.25mm for the 3-part template (see Fig. \ref{templates}). 


Figure \ref{nodefect} shows a blade section without defect. Figure \ref{defect-025mm} presents a deviation of 0.25mm near the leading edge from a 1-part template. The corresponding length of the LE defect is 3.399mm.  All the defects considered in present work are within ISO Class S. 

\

As shown in Fig. \ref{geometry details No defect}, Fig. \ref{geometry details 0.5mm defect}, Fig. \ref{geometry details 0.25mm defect} and Fig. \ref{geometry details 0.1mm defect}, the DTMB modified NACA66 foil sections ($a$=0.8, $t/c$=0.0416, $f/c$=0.014, $c$=1,000mm) with no defect and with three different sizes of defect are investigated in the present work. Details of these sections are summarized in Table   \ref{foils defect geometry}. Note that the chord length, $c$, the maximum thickness ratio, $t/c$, and the maximum camber ratio, $f/c$,  of these four sections were kept the same. 



The sizes of defects are listed in Table  \ref{sizes of defects}, where the point A and point B are the front end and the back end of the LE defect, respectively. Length is the length of the defect. $\Delta C$ means the distance between point A and B in X-axis direction, and $\% C$ represents the ratio of $\Delta C$ to the chord length.



\section{Coordinate System}

The coordinate system for all 2D simulations is presented in Fig \ref{geometry}. The origin, $O$, is at the leading edge of the foil. The $OX$ axis is from the leading edge to the trailing edge (TE) along the chord line and the $OY$ axis is perpendicular to the chord line.

\section{Non-dimensionalizations and Definition of Variables}

The Reynolds number, $Re$, is defined as:

\begin{equation}
Re = \frac{\rho U L}{\mu}
\label{eq19}
\end{equation}

where $U$ is the flow velocity at infinity, $L$ is the chord length, and $\mu$ is the dynamic viscosity of water. 

The negative minimum pressure coefficient, $-C_{p_{\rm min}}$, is defined as:

\begin{equation}
-C_{p_{\rm min}} = -{\rm min}(C_p)
\label{eq20}
\end{equation}

where $C_p$ is the pressure coefficient, given as:

\begin{equation}
C_p = \frac{p-p_a-\rho gh}{\frac{1}{2}\rho U^2}
\label{eq21}
\end{equation}

where $\rho$ and $g$ denote the density of water and the gravitational acceleration, respectively, $p_a$ is the air pressure, and $\rho gh$ is the hydrostatic pressure.

The cavitation number, $\sigma$, is defined as:

\begin{equation}
\sigma = \frac{p_\infty-p_v}{\frac{1}{2}\rho U^2}
\label{eq22}
\end{equation}

where $p_\infty$ is the pressure in far field and $p_v$ is the vapor pressure of water. 

The lift coefficient, $C_l$, and the drag coefficient, $C_d$, for the foil are defined as: 

\begin{equation}
C_l = \frac{F_l}{\frac{1}{2}\rho U^2 A}
\label{eq23}
\end{equation}

\begin{equation}
C_d = \frac{F_d}{\frac{1}{2}\rho U^2 A}
\label{eq23-1}
\end{equation}

where $F_l$ and $F_d$ are the lift and the drag, respectively, $A$ is the characteristic area, which was set as 1.0 in the present 2-D studies. 


\section{Computational Domain}

The computational domain must be sufficiently large to represent the infinite fluid domain. It is preferable to use structured grids for the simulations for more accurate solutions. The geometry of the domain should be chosen in such a way that the generated structured grids are in high quality. To generate the grids for foils with defects, adequate grids must be distributed on the foil surface, especially near the defects, to resolve the flow details. On the other hand, since a large computational domain is required,  the grid spacing needs to be increased when approaching to the domain boundaries for the purpose of computing efficiency. These lead to some challenges in generating structured grids.  

In the present work, various domain and grid topologies were investigated, including rectangular and circular domains, with an objective to compare the grid quality and hence the convergence of the RANS solutions. 

The rectangular domains with H-type, O-type and C-type grid topologies, and a circular domain are shown in Figs.  \ref{Intro_Topology_Rectangle_H}, \ref{Intro_Topology_Rectangle_O}, \ref{Intro_Topology_Rectangle_C} and \ref{Boundary conditions for Circular O-type grids}, respectively.



The qualities of grids for the foil with 0.5mm LE defect at the angle of attack of $\alpha=4^\circ$ are compared in Table  \ref{grid quality} in terms of maximum included angle and maximum non-orthogonality. The maximum include angle is used to measure the grid skewness. The non-orthogonality is defined by the angle between the line connecting the centroids of two grids and the normal of the shared edge between the two grids. The larger their values are, the worse the qualities of grids are.

Studies in this work have shown that the circular domain leads to grids in greater quality. Therefore, the circular domain with the O-type topology was employed. 


\section{Definition of $y^+$, Aspect Ratio and Stretching Ratio}

The generation of structured grids is dependent on the specified $y^+$, the grid aspect ratio (AR), and the grid stretching ratio (SR). The non-dimensional  first-grid spacing, $y^+$, is estimated by:
\begin{equation}
y^+ = \sqrt{\frac{0.026 U^2}{2Re^{1/7}}} \frac{\rho \Delta S}{ \mu}
\label{eq24}
\end{equation}
where $\Delta S$ is the height of the first grid near the wall. Note that  $\Delta S$ is measured from the center of the grid cell in Star-CCM+.

The grid aspect ratio is defined as the maximum ratio of grid width to height for 2D grids. As shown in Fig.  \ref{Definition of the aspect ratio and stretching ratio of grid}, the AR of the $n^{\rm th}$ grid is determined by:
\begin{equation}
AR = h_n/w_n
\label{eq25}
\end{equation}
where $w_n$ and $h_n$ are the grid width and the grid height, respectively.

The grid stretching ratio is defined as the ratio of the heights of adjacent cells. As shown in Fig.  \ref{Definition of the aspect ratio and stretching ratio of grid}, the SR of the $n^{\rm th}$ grid is given as:
\begin{equation}
SR = h_{n+1}/h_n
\label{eq26}
\end{equation}
where $h_n$ and $h_{n+1}$ are the heights of the $n^{\rm th}$ and the $(n+1)^{\rm th}$ grids, respectively.

\section{Grid Distribution on the Foil Surface}



The grids on the foil surface are generated based on $y^+$, AR and grid distribution. As shown in Fig. \ref{Grid distribution on the surface}, the suction and the pressure sides are both divided into three segments. Uniform grids are distributed on the leading and the trailing edge segments while non-uniform grids are generated in the middle segment. 

As an example, Fig. \ref{surfacegrid} shows grids near the leading edge and the tailing edge for the foil with 0.5mm LE defect. For this case, the total number of grids on the foil surface is 13,695, including 1,909/1,885 on the suction/pressure side of the leading edge segment, 3,796/3,770 on the suction/pressure side of the trailing edge segment, and 1,168/1,167 on the suction/pressure side of the middle segment. The first-grid spacing, $y^+$, is 1.0 in this example. The corresponding grid aspect ratios on the leading edge, the trailing edge and the middle segments are 40, 20 and 300, respectively. Note that 52 grids  distributed over the 0.5mm defect to resolve the flow details. Uniform grids with AR=20 were distributed near the trailing edge to improve the simulation of vortex flow. 



\section{Boundary Conditions}

Boundaries conditions for the circular domain are presented in Fig.  \ref{Boundary conditions for Circular O-type grids}. Note that the hydrostatic pressure was not taken into account in the present simulations. The pressure boundary condition with $p=0$ was specified on the outlet. A no-slip wall boundary condition was imposed on the surface of the foil section. The Reynolds number for all cases was $Re=3\times 10^7$. At the inlet boundary, an uniform velocity of $U$ = 30 m/s was specified. 


\section{Convergence Criteria}

Two levels of convergence criteria were applied in the present studies, including:

\begin{itemize}

\item Residuals, defined as normalized root-mean-squared values in Star-CCM+, are used as the first convergence criterion. Three-order-of-magnitude reduction in residuals is considered as an acceptable level. Note that residuals are not the only measure for convergence. The initial values strongly influence the residuals. If the initial solution satisfies the discretized equations very well, the residuals would not reduce  significantly. Therefore, it is necessary to examine the convergence of lift and drag coefficients as well as the minimum pressure coefficient. 

\item For the convergence of lift, drag and pressure coefficients, the changes between their values at the present and previous iterations are used as indicators after the residual criteria are satisfied. For the lift and drag coefficients, it is considered acceptable if the changes between two iterations are in the order of 10$^{-6}$. For the minimum pressure coefficient, the acceptable value is in the order of 10$^{-5}$. 

\end{itemize}

The maximum number of iterations was set as 15,000 in all simulations. Residuals and changes in lift, drag and minimum pressure coefficients were then checked against the convergence criteria described above. 

\clearpage

\section{Simulation Parameters and Cases}

The air pressure is set as $p_a=101,325\rm Pa$. The density of water is $\rho=1.0\times 10^3\rm kg/m^3$ and the kinematic viscosity of water is $1.0\times 10^{-6}\rm m^2/s$. A total of 1,020 cases at various angles of attack were simulated with different turbulence models by changing the first-grid spacing, the grid aspect ratio, the grid stretching ratio, and the number of grids near the LE and the defect.

A summary of the convergence studies on the circular computational domain is provided below:

\begin{itemize}
\item Domain sizes in term of radius of domain: 6m, 12m, 18m, 24m, 30m and 36m. 

\item Grid stretching ratios: 1.1 and 1.2.

\item Grid aspect ratios at LE: 320.0, 160.0, 113.12, 80.0, 56.56 and 40.0.

\item Grid aspect ratios at TE: 120.0, 80.0, 60.0, 40.0, 30.0 and 20.0.

\item First-grid spacing, $y^+$: 1.0, 1.414, 2.0, 2.828, 4.0, 5.0, 10.0, 15.0, 30.0, 60.0, 90.0 and 120.0.

\item Turbulence models: Spalart-Allmaras one-equation model; $k-\epsilon$, $k-\omega$ and SST $k-\omega$ two-equation models; and elliptic blending and linear pressure-strain Reynolds stress models.

\end{itemize}

In these convergence studies, the number of grids ranges from 791,415 to 2,013,312.

\section{Summary of Best-Practice Settings}

Based on the extensive convergence studies, best-practice settings for 2-D simulations with the Star-CCM+ steady RANS solver are determined and presented in Table \ref{Best-Practice Settings}. The numbers of grids for the foil without/with defect are summarized in Table \ref{Number of grids}. Other default settings for the solver are summarized in Table  \ref{Default settings}. 





% \begin{table}[!htp]
% \setlength{\abovecaptionskip}{-5pt}
% \caption{Default settings used in the present simulations with Star-CCM+}
% \begin{center}
% \label{Default settings}
% \begin{tabular}{ll}
% \hline
% Simulation Parameters  					         &  Default Settings 				  \\
% \hline        
% Convection scheme      				             &  $2^{nd}$-order upwind 			  \\
% Gradient method        					         &  Hybrid Gauss-Least squares method \\
% Limiter method         					         &  Venkatakrishnan method 			  \\
% Custom accuracy level selector 			         &  $2^{nd}$-order 				      \\
% Reference pressure     					         &  101,325 Pa                        \\
% Initial turbulence intensity, $I$                &  1\%  							  \\
% Initial turbulent viscosity ratio, $\mu_t/\mu$   &  10.0 							  \\
% Linear solver                                    &  Algebraic multigrid methods (AMG) \\
% Relaxation scheme                                &  Gauss-Seidel 					  \\
% Under-relaxation factor for velocity             &  0.4 							  \\
% Under-relaxation factor for pressure             &  0.1 							  \\
% Under-relaxation factor for turbulence           &  0.8 							  \\
% Convergence tolerance 							 &  0.1 						      \\
% \hline
% \end{tabular}
% \end{center}
% \end{table}



\section{Simulation Results with Best-Practice Settings}

\subsection{Cavitation Buckets for the Modified NACA66 Section ($a$=0.8, $t/c$=0.2, $f/c$=0.02) without Defect}

The predicted cavitation buckets of the modified NACA66 section ($a$=0.8, $t/c$=0.2, $f/c$=0.02) without defect in terms of the negative minimum pressure coefficient  at angles of attack from $-5^{\circ}$ to $6^{\circ}$ are shown in Fig.  \ref{Cavitation buckets potential flow} and compared with  the potential-flow solutions by Brockett (1966) and the numerical results with ANSYS CFX by Hally (2018). The agreement is in general good.




\subsection{Cavitation Buckets for the Modified NACA66 Foils ($a$=0.8, $t/c$=0.0416, $f/c$=0.014) without and with LE Defect}

Cavitation buckets in terms of the negative minimum pressure coefficient were compared with the results by Hally (2018) for the foil ($a$=0.8, $t/c$=0.0416, $f/c$=0.014) without defect and three of them with different LE defects. Table ~\ref{Operating conditions} lists the details of simulation cases. 


As shown in Fig.  \ref{Cavitation buckets}, the cavitation buckets are narrowed by the defects at the LE in the region of typical propeller design. In other words, the incipient cavitation speed is reduced by the LE defect. 


As examples, the minimum pressure coefficient and its location on the foil surface, residuals of simulations, and the convergence of drag and lift coefficients at the angle of attack of 0.8$^\circ$ are presented in the following sections.  


\subsection{Pressure Contours and Streamlines}


The contours of pressure coefficient and streamlines near the LE at $\alpha=0.8^\circ$ for foils with no defect, 0.5mm defect, 0.25mm defect and 0.1mm defect are presented in Fig.  \ref{Contour buckets 0.8}. It can be observed that the defect led to lower pressure near the LE.  Although locations of the minimum pressure depend on the size of a defect, they are all allocated close to the upper end of the flat defect. For example, the upper end point of the 0.5mm defect is (0.00320m, 0.00266m) and the location of the minimum pressure is at (0.00323m, 0.00267m). 

Table \ref{inception speed} presents the negative minimum pressure coefficient, its location and the corresponding cavitation inception speed for the foils with no defect, 0.5mm defect, 0.25mm defect and 0.1mm defect at $\alpha=0.8^\circ$. 

Note that negative pressure coefficients and their locations along with lift and drag coefficients are provided in Appendix A at various angles of attack. They were all computed using the best-practice settings. 

Denoting the inception speeds for the foils without defect and with defect as $U_0$ and $U'$, respectively, the inception speed ratio is defined as 

\begin{equation}
\eta = \frac{U'}{U_0} = \sqrt{\frac{C_{p_{\rm min}}}{C'_{p_{\rm min}}}}
\end{equation}

where $C_{p_{\rm min}}$ and $C'_{p_{\rm min}}$ are the minimum pressure coefficients for the foils without and with defect, respectively. The inception speed ratios for the foils with different sizes of defect are given in the table along with the reduction of inception speed in percentage with respect to that of the foil without defect (as designed). It can be seen that even the smallest defect leads to a significant reduction in the cavitation inception speed at angle of attack of 0.8$^\circ$. 


The contours of pressure coefficient and streamlines at all angles of attack $-4^\circ < \alpha < 4^\circ$ are presented in Fig. ~\ref{Contour buckets 4.0} to Fig. ~\ref{Contour buckets -4.0}.

%\include{Final-Report-Part-CpContour}

\subsection{Pressure Plot}

The pressure distributions near the LE at $\alpha=0.8^\circ$ for the four foils, i.e., with no defect, 0.5mm defect, 0.25mm defect and 0.1mm defect, are shown in Fig.  \ref{Cp distribution buckets 0.8}. It can be seen that the pressures on the suction side were significantly changed by the defect near LE. The pressure distributions at all angles of attack $-4^\circ < \alpha < 4^\circ$ are shown in Fig. ~\ref{Cp distribution buckets 4.0} to Fig. ~\ref{Cp distribution buckets -4.0}.




\subsection{Residuals, -$C_{p_{\rm min}}$, $C_d$ and $C_l$}

Residuals of simulations for the four foils with no defect, 0.5mm defect, 0.25mm defect and 0.1mm defect at $\alpha=0.8^\circ$ are shown in Figs.  \ref{residuals buckets No defects P0p80}, \ref{residuals buckets 0.5 mm defect P0p80}, \ref{residuals buckets 0.25 mm defect P0p80} and \ref{residuals buckets 0.1 mm defect P0p80}, respectively. Three order-of-magnitude reduction in residuals were achieved. In these figures, the legend of ``Continuity" denotes the residual for the continuity equation, ``X-momentum" is the residual for the momentum equation (X-component), ``Y-momentum" is the residual for the momentum equation (Y-component), ``Tke" represents the residual for the transport equation of turbulence kinetic energy ($k$), and ``Sdr" denotes the residual for the transport equation of specific dissipation rate ($\omega$).

The corresponding iteration histories for the drag and lift coefficients are shown in Figs.  \ref{CdCl buckets No defects P0p80}, \ref{CdCl buckets 0.5 mm defect P0p80}, \ref{CdCl buckets 0.25 mm defect P0p80} and \ref{CdCl buckets 0.1 mm defect P0p80}, respectively.

The negative minimum pressure coefficient, -$C_{p_{\rm min}}$, the drag and lift coefficients, $C_d$, $C_l$, and their changes between  two iterations are summarized in Table  \ref{Numerical results best-practice P0p80}. 


The cavitation inception speeds for the typical design range of angle of attack ($-1.5^\circ < \alpha < 2.0^\circ$) were studied. The choice of angle of attack, $\alpha$, was based on the estimation of the hydrodynamic pitch angle, $\beta_i$. The induced velocity of a rotating propeller includes three velocity components, radial, tangential and axial. The radial component of the induced velocity can be ignored based on the assumptions that there is no contraction or reduction in diameter of the slipstream. The other two components have a major influence on the angle of attack although the induced velocity is small as compared to the inflow velocities for a moderately loaded propeller. For estimation, the propeller blade can be considered as a lifting line. The components of the induced velocity at the lifting line are one half the values downstream. It is also assumed that the resultant induced velocity is perpendicular to the resultant inflow velocity. The angle of attack is therefore calculated by the difference of geometrical pitch angle, $\phi$, and the hydrodynamic pitch angle, $\beta_i$, which can be estimated by the advance angle $\beta$ and the minimum energy loss. The results of $\beta$, $\beta_i$, and the corresponding angle of attack, $\alpha$, for the two designed propellers in the work by Eckhardt and Morgan (1955) are presented in Table \ref{choice of alpha}. It can be observed that the typical angle of attack for the section at 0.7R is less than 2 degrees.


Based on the cavitation buckets for the Modified NACA66 Foils ($a$=0.8, $t/c$=0.0416, $f/c$=0.014) as shown in Fig. \ref{Cavitation buckets}, the reduction percentages in inception speed due to LE defects in comparison to that for the foil without defect are presented in Fig. \ref{Reduction in cavitation inception speed}. In the typical design range of angle of attack ($-1.5^\circ < \alpha < 2.0^\circ$), the reduction in reception speed can reach to 60\% for the 0.5mm defect around $\alpha=0.75^\circ$. Between 0 and 1.5$^\circ$, the reduction increases with the size of LE defect. At $1.5^\circ < \alpha < 2^\circ$, the reductions for the three defects are around 15\% to 20\%. 

In summary, the LE defects significantly reduce the cavitation inception speeds at the normal range of angle of attack.


\subsection{Effect of LE Defect on Efficiency}

The efficiency of 2-D section can be measured by the ratio of lift and drag, i.e., $C_l/C_d$. The comparison of $C_l/C_d$ is shown in Fig. \ref{Reduction in efficiency}. The efficiency for the foils at $1.5^\circ<\alpha<2^\circ$ is slightly affected by the defect. For $2^\circ<\alpha<4^\circ$ , the efficiency is reduced with the increase of the defect length.



\chapter{Conclusions}
 
All the LE defects examined are within ISO 484 Class S tolerances (+/-0.5mm for a 1-part template or +/-0.25mm for the 3-part template). The DTMB modified NACA66 (a=0.8 camber) sections without and with LE defects were investigated at various angles of attack with the 2-D steady RANS solver in Star-CCM+ on structured grids.

Convergence studies were first carried out to examine the effects of domain type, domain size, grid distribution, grid resolution, and turbulence model on the solutions. Based on the results of convergence studies, best-practice settings were proposed for simulations of 2-D foils using Star-CCM+. With the best-practice settings, studies were carried out to verify the minimum pressure coefficient envelops of the modified NACA66 foil section ($a$=0.8, $t/c$=0.2, $f/c$=0.02) without defect. Numerical results were generally in good agreement with potential-flow solutions by Brockett (1966) and the RANS solutions with ANSYS CFX by Hally (2018). 

CFD simulations using the best-practice settings were extended to the modified NACA66 foil sections ($a$=0.8, $t/c$=0.0416, $f/c$=0.014) with three different sizes of defects near LE, representing three levels of manufacturing tolerances within Class S. The predicted minimum pressure coefficients for the NACA66 sections without and with LE defects were compared at various angles of attack. Comparative studies showed that the LE manufacturing defects in various sizes within ISO Class S have large effects on the cavitation performance of 2-D foil sections in terms of reduced cavitation inception speed in the typical design range of angle of attack. 

The following conclusions are made from the 2-D studies:

\begin{itemize}

\item Class S defects close to LE narrow the cavitation buckets in the $C_{p_{\rm min}}-\alpha$ space in the typical design range of angle of attack, $-1.5^\circ<\alpha<2^\circ$. As a consequence, such a defective section would experience cavitation at a lower speed than the design one. Smaller defects than Class S maximum deviation show a similar effect.

\item The detrimental cavitation effect seems to be primarily on the side of the section with defect. A defect right on the leading edge ($x$=0 and $y$=0) would affect cavitation on both sides of the section. 

\item The defects can cause pressure drops at the furthest-forward edge of a LE defect. This leads to flow separation at angles of attack at the upper end of normal range of operations. When a section experiences flow separations, the section lift decreases and the drag increases resulting in a reduction in propeller efficiency.

\item The discrepancies between the results from Star-CCM+ and ANSYS CFX at angle of attack of 4 degrees for the modified NACA66 foil sections ($a$=0.8, $t/c$=0.0416, $f/c$=0.014) with 0.5mm defect are due to different CFD methods to assess the flow separations. The prediction of flow near the stalling point remains a challenging issue in CFD simulations.

\end{itemize}


%%\clearpage

\chapter*{References}
%\addcontentsline{toc}{chapter}{References}
\addcontentsline{toc}{section}{References}

AQUO Project no. 314227, WP 2: Noise sources, impact of propeller noise on global URN, Task T2.5, D2.8, Rev. 1, July 16, 2015. 

van Beek, T., Janssen, A., An Integrated design and production concept for ship propellers. 34th WEGEMT at University of Delft, June 2000. 

Brockett, T., 1966, Minimum pressure envelopes for modified NACA-66 sections with NACA a=0.8 camber and BUSHIPS Type I and Type II Sections, David Taylor Model Basin. 

CISMaRT, 2019, Report on the CISMaRT/Transport Canada Workshop on ship noise mitigation technologies for a quieter ocean. 

Eca, L. and Hoekstra, M., 2014, A procedure for the estimation of the numerical uncertainty of CFD calculations based on grid refinement studies. J Computational Physics; 262:104-30.

Eckhardt, M.K. and Morgan, W.B., 1955. A propeller design method (Vol. 63). SNAME.

Gospodnetic, D. and Gospodnetic S., Integrated propeller manufacturing system. Shipbuilding symposium “Sorta”, Zagreb, Croatia, May 1996.

Gospodnetic, S., CNC milling of monoblock propellers to final form and finish. Ottawa Marine Technical Symposium, Ottawa, February 2013.

Gospodnetic, S., CNC machining of propellers to better than class S tolerances. SNAME 14th Propeller \& Shafting Symposium, Norfolk, Virginia, Sept. 2015.

Hally, D., 2018, Preliminary CFD simulations of 2-D foils with defects, DRDC Atlantic Centre, Canada.

ISO 484-1: 2015 (E). Shipbuilding – Ship screw propellers – Manufacturing tolerances – Part 1: Propellers of diameter greater than 2.5m, 2015. 

ISO 484-2: 2015 (E). Shipbuilding – Ship screw propellers – Manufacturing tolerances – Part 2: Propellers of diameter between 0.80 and 2.5m inclusive, 2015. 

Janssen, A. and Leever, S., Propeller manufacture and tolerances, Encyclopedia of Maritime and Offshore Engineering. John Wiley and Sons Ltd. 2017. 

Jones, W. P., and Launder, B. E., 1972, The prediction of laminarization with a two-equation model of turbulence. International Journal of Heat and Mass Transfer, Vol. 15, pp. 301-314.

Menter, F. R., 1993, Zonal two equation $k-\omega$ turbulence models for aerodynamic flows. AIAA Paper 93-2906. In 23rd fluid dynamics, plasmadynamics, and lasers conference.

NAVSEA, Standard propeller drawing, No. 810-4435837, Rev. B, 2004. 

Southall, B.L., Scholik-Schhlomer, A.R., Hatch, L., Brgmann, T., Jasny, M., Metcalf, K., Weilgart, L., and Wright, A.J., Underwater noise from large commercial ships – international collaboration for noise reduction, Encyclopedia of Maritime and Offshore Engineering. John Wiley and Sons Ltd., 2017. 

Spalart, Philippe, and Steven Allmaras., A one-equation turbulence model for aerodynamic flows. 30th aerospace sciences meeting and exhibit. 1992.

Tremblay, C. and Gospodnetic S., Manufacturing propellers in the 21st century. Maritime Engineering Journal, Ottawa, Spring 2017.

Wilcox, D.C., 1988, Re-assessment of the scale-determining equation for advanced turbulence models. AIAA Journal, Vol. 26, No. 11, pp. 1299-1310.



%\include{Final-Report-Appendices-A}

 

\end{document}
