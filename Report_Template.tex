\documentclass[onecolumn,11pt]{report}
\usepackage{longtable,graphicx}
\usepackage{nomencl}
\usepackage{amsbsy}
\usepackage[none]{hyphenat}
\usepackage[firstpage]{draftwatermark}
\SetWatermarkScale{1.0}
\SetWatermarkLightness{0.85}
\usepackage{appendix}

\usepackage{graphicx}

\usepackage{epsfig} %% for loading postscript figures
%\usepackage[outdir=./]{epstopdf}

\usepackage{graphicx}
\usepackage{float}
\usepackage{lipsum} % for dummy text

%\usepackage{titlesec}
%\usepackage{cite}
\usepackage{float}
\usepackage{color}
\usepackage[utf8]{inputenc}
%\usepackage[portuguese]{babel}
\usepackage{makeidx}
\usepackage{enumerate}
\usepackage[official]{eurosym}
\usepackage{indentfirst}

%\usepackage{hyperref}
%make page number, not text, be link on toc,lof,lot
\usepackage[linktocpage]{hyperref}
%Break the long titles, but lose the hyerlink
%\usepackage[breaklinks=true]{hyperref}
%Control the space between number and title
%\usepackage{tocbasic}
%\DeclareTOCStyleEntry[dynnumwidth]{tocline}{figure}
%\DeclareTOCStyleEntry[dynnumwidth]{tocline}{table}

\usepackage{listings}

\usepackage{array}
\usepackage{makecell}
\usepackage{multicol}
\usepackage{multirow}
\usepackage{booktabs}

\usepackage{amsmath}
\usepackage{tabularx}
\usepackage{flexisym} 
\usepackage{placeins} 

\usepackage{textcomp}

%%\usepackage{nopageno}

\usepackage{caption}
%\usepackage{subcaption}

\usepackage{siunitx}

\usepackage{appendix}

\usepackage{subfig}
\usepackage{adjustbox}


\usepackage{tabu}
\usepackage{xcolor}
\usepackage{color, colortbl}

\usepackage{dirtree}

%\usepackage{rotating}

\definecolor{dkgreen}{rgb}{0,0.6,0}
\definecolor{gray}{rgb}{0.5,0.5,0.5}
\definecolor{mauve}{rgb}{0.58,0,0.82}

\lstset{frame=tb,
  language=C,
  aboveskip=3mm,
  belowskip=3mm,
  showstringspaces=false,
  columns=flexible,
  basicstyle={\small\ttfamily},
  numbers=none,
  numberstyle=\tiny\color{gray},
  keywordstyle=\color{blue},
  commentstyle=\color{dkgreen},
  stringstyle=\color{mauve},
  breaklines=true,
  breakatwhitespace=true,
  tabsize=3,
  moredelim=**[is][\color{red}]{@}{@},
  moredelim=**[is][\color{blue}]{^}{^}
}


%\usepackage{tocloft}
%\setlength{\cftfignumwidth}{3.55em}

%\usepackage{tocloft}
%\setlength{\cftfignumwidth}{5em}

%\usepackage{caption} 
%\captionsetup[table]{skip=10pt}
%
%\usepackage{titling}
%\setlength{\columnsep}{10mm}
\makeglossary
% Define some macros:

%--------------------------------------------------------------
%\setlength{\parskip}{2ex plus0.5ex minus0.5ex}
%\setlength{\parindent}{0.0mm}
%\setlength{\textheight}{8.5in}
%\setlength{\textwidth}{6.0in}
%\setlength{\oddsidemargin}{0.5in}
%\setlength{\evensidemargin}{0.0in}
%\setlength{\topmargin}{0.0in}
%\setlength{\arrayrulewidth}{.3mm}
%--------------------------------------------------------------
\parskip .3cm
%\baselineskip 14pt
\parindent 0. in
\textheight=8.5 in
\textwidth=6.5 in
%\hoffset -0.5in
%\voffset -0.90in
\voffset .0 in
\oddsidemargin 0. in \evensidemargin 0 in \topmargin 0.0in
\renewcommand{\baselinestretch}{1}
%-------------------------------------------------------


\newcommand{\mychapter}[2]{
    \setcounter{chapter}{#1}
    \setcounter{section}{0}
    \chapter*{#2}
    \addcontentsline{toc}{chapter}{#2}
}

%----------------------------------------------------------------------------------------

\begin{document}

%%\maketitle % Print the title


\begin{titlepage}

~~
\vspace{1cm}
\centering
{\Large\bfseries {Title Here}}

\vspace{2cm}

{\large {\textit {\bf Progress Report - 1}}}

\vspace{2cm}

{\large  xxxxxxAuthorsxxxxx}

\vspace{0.5cm}

{\large Department of Ocean and Naval Architectural Engineering \\ Memorial University}


\vspace{2cm}


{\large \underline{Submitted to}}

\vspace{1cm}

{\large Dr. XXX}

\vspace{0.5cm}

{\large XXXXX}

\vspace{3cm}


{\large September 23, 2019}

\clearpage

\tableofcontents

\listoffigures
 
\listoftables




\makenomenclature
\mbox{}
\nomenclature{LE}{Leading edge}

\printnomenclature

\end{titlepage}



%----------------------------------------------------------------------------------------
%	INTRODUCTION
%----------------------------------------------------------------------------------------


\clearpage

\section*{\centering Executive Summary}
\addcontentsline{toc}{section}{Executive Summary}
 
Propellers are 

\begin{itemize}

\item xxx

\end{itemize}



\newpage

\chapter{Introduction}

Underwater radiated noise (URN) 


\section{Scope of the Present Work}

The objective of this project 

%----------------------------------------------------------------------------------------%

\chapter{Theoretical Background}

\section{Governing Equations}

The governing RANS equations for the incompressible viscous flow are:
\begin{equation}
\frac{\partial u_i}{\partial x_i} = 0
\label{eq1}
\end{equation}  

 
%----------------------------------------------------------------------------------------%
 

\chapter{Numerical Simulations}

Numerical simulations were carried out 





% \begin{figure} [hpbt]
% \centerline{\psfig{figure=./figures/titlepage.png,width=0.4\columnwidth}}
% \caption{\hspace{0.04cm} Coordinate Systems}
% \label{CoordinateSystem}
% \end{figure}

% \begin{table}[!htp]
% \setlength{\abovecaptionskip}{-5pt}
% \caption{Default settings used in the present simulations with Star-CCM+}
% \begin{center}
% \label{Default settings}
% \begin{tabular}{ll}
% \hline
% Simulation Parameters  					         &  Default Settings 				  \\
% \hline        
% Convection scheme      				             &  $2^{nd}$-order upwind 			  \\
% Gradient method        					         &  Hybrid Gauss-Least squares method \\
% Limiter method         					         &  Venkatakrishnan method 			  \\
% Custom accuracy level selector 			         &  $2^{nd}$-order 				      \\
% Reference pressure     					         &  101,325 Pa                        \\
% Initial turbulence intensity, $I$                &  1\%  							  \\
% Initial turbulent viscosity ratio, $\mu_t/\mu$   &  10.0 							  \\
% Linear solver                                    &  Algebraic multigrid methods (AMG) \\
% Relaxation scheme                                &  Gauss-Seidel 					  \\
% Under-relaxation factor for velocity             &  0.4 							  \\
% Under-relaxation factor for pressure             &  0.1 							  \\
% Under-relaxation factor for turbulence           &  0.8 							  \\
% Convergence tolerance 							 &  0.1 						      \\
% \hline
% \end{tabular}
% \end{center}
% \end{table}



\section{Simulation Results with Best-Practice Settings}

\chapter{Conclusions}
 
All 


%%\clearpage

\chapter*{References}
%\addcontentsline{toc}{chapter}{References}
\addcontentsline{toc}{section}{References}

Brockett, T., 1966, Minimum pressure envelopes for modified NACA-66 sections with NACA a=0.8 camber and BUSHIPS Type I and Type II Sections, David Taylor Model Basin. 

Eca, L. and Hoekstra, M., 2014, A procedure for the estimation of the numerical uncertainty of CFD calculations based on grid refinement studies. J Computational Physics; 262:104-30.


%\include{Final-Report-Appendices-A}


\end{document}
